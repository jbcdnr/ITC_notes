\documentclass{article}
% General document formatting
\usepackage[margin=1in]{geometry}
\usepackage{amsmath}
\usepackage{amssymb}
\usepackage{amsfonts}
\usepackage{amsthm}
\usepackage{mathrsfs}
\usepackage{cleveref}
\usepackage{xcolor}
\usepackage[toc,page]{appendix}

\title{Information Theory and Coding - Prof.~Emere Telatar}
\date{\today}
\author{Jean-Baptiste Cordonnier, Sebastien Speierer, Thomas Batschelet}

% define example environments
% \newcounter{example}[section]
% \newenvironment{example}
%     {
%     \refstepcounter{example}
%     \begin{center}
% 	    \begin{tabular}{|p{0.8\textwidth}|}
% 		    \hline\\
% 			\textbf{Example~\theexample.}\\
% 		    }
% 		    {
% 		    \\\hline
% 	    \end{tabular}
%     \end{center}
%     }

% define definition environments
% \newcounter{definition}[section]
% \newenvironment{definition}
%     {
%     \refstepcounter{definition}
%     \begin{center}
% 	    \begin{tabular}{|p{0.8\textwidth}|}
% 		    \hline\\
% 			\textbf{Definition~\thedefinition.}\\
% 		    }
% 		    {
% 		    \\\hline
% 	    \end{tabular}
%     \end{center}
%     }

\newtheorem{theorem}{Theorem}[section]
\newtheorem{corollary}{Corollary}[theorem]
\newtheorem{claim}{Claim}[section]
\newtheorem{observation}{Observation}[section]
\newtheorem{lemma}[theorem]{Lemma}
\newtheorem{definition}{Definition}[section]
\newtheorem{proposition}{Proposition}[section]
\newtheorem{notation}{Notation}
\newtheorem{example}{Example}

\newcommand{\todo}[1]{\textcolor{red}{TODO: #1}}
\renewcommand{\Pr}[1]{Pr\left\{#1\right\}}
\newcommand{\Ex}[1]{E\left[#1\right]}
\newcommand{\pfrac}[2]{\left( \frac{#1}{#2} \right)}

% math bold
\def\*#1{\mathbf{#1}}
% manuscript characters
\def\D{\mathcal{D}}
\def\V{\mathcal{V}}
\def\L{\mathcal{L}}
\def\U{\mathcal{U}}
\def\N{\mathcal{N}}
\def\X{\mathcal{X}}

\def\C{\mathscr{C}}

\DeclareMathOperator{\E}{\mathbb{E}}% expected value


\begin{document}

\maketitle

\section{Data compression}

\begin{definition}[Information]
  Abstractly, \textbf{information} can be thought of as the resolution of uncertainty.
\end{definition}

Given an alphabet $\U$ (e.g. $\U = \{a, ..., z, A, ..., Z, ...\}$), we want to assign binary sequences to elements of $\U$, i.e.

\begin{align*}
	\C: \U \rightarrow \{0, 1\}^* = \{\emptyset, 0, 1, 00, 01, ...\}
\end{align*}

For $\X$ a set

\begin{align*}
	\X^n &\equiv \{ (x_0 ... x_n), x_i \in \X\} \\
	\X^* &\equiv \bigcup_{n \geq 0} \X^n
\end{align*}

\begin{definition}
	A code $\C$ is called \textbf{singular} if
	\begin{align*}
		\exists (u, v) \in \U^2, u \neq v \quad s.t. \quad C(u) = C(v)
	\end{align*}
	Non singular code is defined as opposite
\end{definition}

\begin{definition}
	A code $\C$ is called \textbf{uniquily decodable} if
	\begin{align*}
		\forall u_1,...,u_n,v_1,...,v_n \in \U^* \quad s.t. \quad u_1,...,u_n \neq v_1,...,v_n
	\end{align*}
	we have
	\begin{align*}
		\C(u_1)...\C(u_n) \neq \C(v_1)...\C(v_n)
	\end{align*}
	i.e, $\C$ is non-singular
\end{definition}

\begin{definition}
	Suppose $\C : \U \rightarrow \{ 0, 1\}^*$ and $\D : \V \rightarrow \{ 0, 1\}^*$ we can define

	\begin{align*}
		\C \times \D : \U \times \V \rightarrow \{0, 1\}^*
		\quad \text{ as } \quad
		(\C \times \D)(u, v) \rightarrow \C(u)\D(v)
	\end{align*}
\end{definition}

\begin{definition}
	Given $\C : \U \rightarrow \{ 0, 1\}^*$, define
	\begin{align*}
		\C^* : \U^* \rightarrow \{0, 1\}^*
		\quad \text{ as } \quad
		\C^*(u_1, u_n) = \C(u_1)...\C(u_n)
	\end{align*}
\end{definition}

\begin{definition}
	A code $\U \rightarrow \{0, 1\}^*$ is \textbf{prefix-free} is for no $u \neq v$ $\C(u)$ is a prefix of $\C(v)$.
\end{definition}

\begin{theorem}
	If $\C$ is prefix-free then $\C$ is uniquely decodable.
\end{theorem}

\begin{definition}
  $l(\C(u))$ is the length of the code word $\C(u)$ and $l(\C)$ is the expected length of the code:
  \begin{align*}
    l(\C) = \sum_u l(\C(u)) p(u)
  \end{align*}
\end{definition}

\begin{definition}[Kraft sum]
  Given $\C : \U \rightarrow \{ 0, 1\}^*$
	\begin{align*}
		kraftsum(\C) = \sum_u 2^{l(\C(u))}
	\end{align*}
\end{definition}

\begin{lemma}
	if $\C : \U \rightarrow \{ 0, 1\}^*$ and $\D : \V \rightarrow \{ 0, 1\}^*$
	then
  $$kraftsum(\C \times \D) = kraftsum(\C) \times kraftsum(\D)$$
	\begin{proof}
    \begin{align*}
      kraftsum(\C \times \D) &= \sum_{u, v} 2^{-(l(\C) * l(\D))} \\
      &= \sum_u 2^{-l(\C)} \sum_v 2^{- l(\D)}
    \end{align*}
	\end{proof}
\end{lemma}

\begin{corollary}
  $kraftsum(\C^n) = (kraftsum(\C))^n$
\end{corollary}

\begin{proposition}
  if $\C$ is non-singular, then
  \begin{align*}
    kraftsum(\C) \leq 1 + \max_n l(\C(u))
  \end{align*}
\end{proposition}

In coding theory, the \textbf{Kraft-McMillan inequality} gives a necessary and sufficient condition for the existence of a uniquely decodable code for a given set of codeword lengths.
\begin{theorem}
  if $\C$ is uniquely decodable, then $kraftsum(\C) \leq 1$
\end{theorem}

\begin{proof}
  $\C$ is uniquely decodable $\equiv$ $\C^*$ is non singular
  \begin{align*}
    &\Rightarrow kraftsum(\C^n) \leq 1 + \max _{u_1, ..., u_n} l(\C^n) \\
    &\Rightarrow kraftsum(\C)^n \leq 1 + n L, \quad L = \max l(\C(n))
  \end{align*}
  A growing exp cannot be bounded by a linear function
  \begin{align*}
    \Rightarrow kraftsum(\C) \leq 1
  \end{align*}
\end{proof}

\begin{theorem}
  Suppose $\C : \U \rightarrow \N$ is such that $\sum_u i^{\C(u)} \leq 1$, then, there exist a prefix-free code $\C: \U \rightarrow \{0, 1\}$ s.t. $\forall l(\C(u)) = \C(u)$
\end{theorem}

\begin{proof}
  Let $\U = \{u_1, ..., u_n\}$ and $\C(u_1) \leq \C(u_2) \leq ... \leq \C(u_k) = \C_{max}$.
  Consider the complete binary tree up to depth $\C_{max}$ initially all nodes are available to be used as codewords.
  For $i = 1, 2, ..., n$, place $\C(u_i)$ at an available node at level $\C(u_i)$ remove all descendant of $\C(u_i)$ from the available list.

  \begin{corollary}
    Suppose $\C: \U \rightarrow \{0, 1\}^*$ is u.d., then there exist an $\C': \U \rightarrow \{0, 1\}^*$ which is prefix-free and $l(\C'(n)) = l(\C(n))$
  \end{corollary}
\end{proof}

\begin{example}
  $\U = \{a, b, c, d\}$, $\C: \{0, 01, 011, 111\}$ and $\C': \{0, 10, 110, 111\}$\\
  In this case, decoding $\C$ may require delay, while decoding $\C'$ is instanteneous.
\end{example}

% -------------------------------------------------------------
\newpage
\section{Alphabet with statistics}

Suppose we have an alphabet $\U$, and suppose we have a random variable $\U$ taking values in $\U$. We denote by $p(u) = Pr(U = u), u \in \U$ with $p(u) \geq 0$ and $\sum_u p(u) = 1$.\\

Suppose we have a code $\C: \U \rightarrow \{0, 1\}^*$. We then have $\C(u)$ a random binary string and $l(\C(u))$ a random integer.

\begin{example}
  $\U = \{a, b, c, d\}$\\
  $p: \{0.5, 0.25, 0.125, 0.125\}$ \\
  $\C: \{0, 01, 110, 111\}$ \\

  then we have
  \begin{align*}
    l(\C(u)) =
    \left\{
      \begin{array}{l}
        1, \quad p = 0.5 \\
        2, \quad p = 0.25 \\
        3, \quad p = 0.125 + 0.125 + 0.25
      \end{array}
    \right.
  \end{align*}
\end{example}

We can measure how efficient $\C$ represents $\U$ by considering
\begin{align*}
  E[l(\C(u))] = \sum_u p(u) \C(u) \quad \text{with} \quad \C(u) = l(\C(u))
\end{align*}

\begin{theorem}
  if $\C$ is u.d., then
  \begin{align*}
    E[l(\C(u))] \geq \sum_u p(u) \log(\frac{1}{p(u)})
  \end{align*}
\end{theorem}

\begin{proof}
  let $\C(u) = l(\C(u))$, we know $\sum_u 2^{-\C(u)} \leq 1$ because $\C$ is u.d.

  \begin{align*}
    E[l(\C(u))] &= \sum_u p(u) \C(u) = \sum_u p(u) \log_2(\frac{1}{q(u)}) \\
    &\equiv \sum_u p(u) \log(\frac{q(u)}{p(u)}) \leq 0 \\
    &\equiv \sum_u p(u) \ln(\frac{q(u)}{p(u)}) \leq 0 \\
    &\leq \sum_u p(u) [\frac{q(u)}{p(u)} - 1]
    = \underbrace{\sum_u q(u)}_{\leq 1} - \underbrace{\sum_u p(u)}_{=1} \leq 0
  \end{align*}
\end{proof}

\begin{theorem}
  For any $\U$, there exists a prefix-free code $\C$ s.t.
  \begin{align*}
    E[l(\C(u))] < 1 + \sum_{u \in \U} p(u) \log(\frac{1}{p(u)})
  \end{align*}
\end{theorem}
\begin{proof}
  Given $\U$, let
  \begin{align*}
    &\C(u) = [\log(\frac{1}{p(u)})] < 1 + \log(\frac{1}{p(u)}) \\
    &\Rightarrow \sum_u 2^{-\C(u)} \leq \sum_u p(u) = 1 \\
    &\Rightarrow \sum_u p(u) \C(u) < \sum_u p(u) \log(\frac{1}{p(u)}) + \underbrace{1}_{\sum p(u)}
  \end{align*}
\end{proof}

\begin{definition}[Entropy]
  Entropy quantifies the amount of uncertainty involved in the value of a random variable or the outcome of a random process.
\end{definition}

\begin{theorem}
  The entropy of a random variable $U \in \U$ is
  \begin{align*}
    H(U) = \sum_{u \in \U} p(u) \log(\frac{1}{p(u)})
  \end{align*}
  with $p(u) = Pr(U = u)$
\end{theorem}
Note that $H(U)$ is a fonction of the distribution $\C_u(.)$ of the random variable $U$, it isn't a function of $U$.

\begin{align*}
  H(U) = E[f(U)] \quad \text{where} \quad f(U) = \log(\frac{1}{p(u)})
\end{align*}

How to design optimal codes (in the sense of minimizing $E[l(\C(u))]$)? \\
Formally, given a random variable $U$, find $\C(u) \rightarrow \N$ s.t.
\begin{align*}
  \sum_{u \in U} 2^{\C(u)} \leq 1
\quad \text{that minimizes} \quad
  \sum_{u \in U} p(u)\C(u)
\end{align*}

Properties of optimal prefix-free codes
\begin{itemize}
  \item if $p(u) < p(v)$ then $\C(u) \geq \C(v)$
  \item The two longest codewords have the same length
  \item There is an optimal code such that the two least probable letters are assigned codewords that differ in the last bit.
\end{itemize}

Observe that if $\{\C(u_1), ... , \C(u_{k-1}), \C(u_k)\}$ is a prefix-free collection of the property that
\begin{align*}
\left.
\begin{array}{l l}
  \C(u_{k-1}) &= \alpha 0 \\
  \C(u_k)     &= \alpha 1
\end{array}
\right.
\quad \text{with} \quad \alpha \in \{ 0, 1\}^*
\end{align*}

then $\{\C(u_1), ..., \C(u_{k-2}, \alpha\}$ is also a prefix-free collection.

Also
\begin{align*}
  \sum_{u \in \U} p(u) l(\C(u)) &= p(u_1) l(\C(u_1)) + ... +  p(u_{k-2}) l(\C(u_{k-2}))
  + [p(u_{k-1}) + p(u_k)](l(\alpha) + 1) \\
  &= (p(u_{k-1}) + p(u_k)) + \sum_{v \in \V} p(v) l(\C'(v))
\end{align*}

So we have shown that with
\begin{align*}
  E[l(\C(U)] = p(u_{k-1}) + p(u_k) + E[l(\C'(v))]
\end{align*}
if $\C$ is optimal for $U$, then $\C'$ is optimal for $V$

% -------------------------------------------------------------
\newpage
\section{Entropy and mutual information}
\label{sec:entropy}

\begin{definition}[Joint entropy]
  Suppose $U, V$ are random variables with $p(u,v) = P(U=u, V=v)$, the joint entropy is
  \[
    H(UV) = \sum_{u,v} p(u,v) \log \frac 1 {p(u,v)}
  \]
\end{definition}

\begin{theorem}
  \[
    H(UV) \leq H(U) + H(V)
  \]
  with equality iff $U$ and $V$ are independants.
\end{theorem}

\begin{proof}
  We want to show that
  \begin{align*}
    \sum_{u,v} p(u,v) \log \frac 1 {p(u,v)} \leq \sum_u p(u) \log \frac 1 {p(u)} + \sum_v p(v) \log \frac 1 {p(v)}
    \iff \sum_{u,v} p(u,v) \log \frac {p(u)p(v)} {p(u,v)} \leq 0
  \end{align*}
  We use $\ln z \leq z - 1~\forall z$ (with equality iff $z=1$):
  \begin{align*}
    \sum_{u,v} p(u,v) \log \frac {p(u)p(v)} {p(u,v)} \leq \sum_{u,v} p(u,v) \left[ \frac {p(u)p(v)} {p(u, v)} - 1 \right] = \sum_{u,v} p(u)p(v) - \sum_{u,v} p(u,v) = 1 - 1 = 0
  \end{align*}
\end{proof}

Same definitions of entropy holds for $n$ symbols.

\begin{definition}[Joint Entropy]
  Suppose $U_1, U_2, \dots, U_n$ are RVs and we are given $p(u_1 \dots u_n)$, the joint entropy is
  \[
    H(U_1, \dots, U_n) = \sum_{u_1 \dots u_n} p(u_1 \dots u_n) \log \frac 1 {p(u_1 \dots u_n)}
  \]
\end{definition}

\begin{theorem}
  \[
    H(U_1, \dots, U_n) \leq \sum_{i=1}^n H(U_i)
  \]
  with equality iff $U$s are independants
\end{theorem}

\begin{corollary}
  if $U_1, \dots, U_n$ are i.i.d. then $H(U_1 \dots U_n) = nH(U_1)$
\end{corollary}

\begin{definition}[Conditional entropy]
  \[
    H(U|V) = \sum_{u,v} p(u,v) \log \frac 1 {p(u|v)}
  \]
\end{definition}

\begin{theorem}
  \[
    H(UV) = H(U) + H(V|U) = H(V) + H(U|V)
  \]
\end{theorem}

\begin{theorem}
  \[
    H(U) + H(V) \geq H(U, V) = H(V) + H(U|V)
  \]
\end{theorem}

\begin{definition}[Mutual information]
  Mutual information measures the amount of information that can be obtained about one random variable by observing another.
  \begin{align*}
  I(U;V) = I(V;U) &= H(U) - H(U|V)\\
  &= H(V) - H(V|U)\\
  &= H(U) + H(V) - H(UV)
  \end{align*}
\end{definition}

We can apply the chain rule on the entropy as follow

\[
  H(U_1, U_2, \dots U_n) = H(U_1) + H(U_2|U_1) + \dots + H(U_n|U_1,U_2 \dots U_{n-1})
\]

\begin{definition}[Conditional mutual information]
  \begin{align*}
    I(U;V|W) &= H(U|W) - H(U|VW)\\
    &= H(V|W) - H(V|UW)\\
    &= \E_{u,v,w} \left[ \log \frac {p(uv|w)} {p(u|w)p(v|w)} \right]
  \end{align*}
\end{definition}

\begin{theorem}
  \[
    I(V;U_1\dots U_n) = I(V;U_1) + I(V;U_2 | U_1) + \dots + I(V;U_n|U_1 \dots U_{n-1})
  \]
\end{theorem}

\begin{notation}
  \[
    U^n \triangleq (U_1, U_2, \dots U_n)
  \]
\end{notation}

\begin{theorem}
  \[
    I(U;V |W) \geq 0
  \]
  equality iff conditioned on $w$, $u$ and $v$ are independant, that is iff $U-V-W$ is a Markov chain.
\end{theorem}

\begin{proof}
  \begin{align*}
    I(U;V|W) &= \frac 1 {\ln 2} \sum_{u,v,w} p(u,v,w) \ln \frac {p(u|w)p(v|w)} {p(uv | w)}\\
    &\geq \frac 1 {\ln 2} \sum_{u,v,w} p(u,v,w) \left[ \frac {p(u|w)p(v|w)} {p(uv | w)} - 1 \right]\\
    &=\frac 1 {\ln 2} \sum_{u,v,w}(p(w)p(u|w)p(v|w) - p(uvw))\\
    &= \frac 1 {\ln 2}(1 - 1) \\
    &=0
  \end{align*}
\end{proof}

% -----------------------------------------------------------------------
\newpage
\section{Data processing}

\begin{theorem}
  $U-V-W$ is a MC $\iff I(U;W|V) = 0$
\end{theorem}

\begin{corollary}
  $I(U;V) \geq I(U;W)$ and by symetry of MC $I(W;V) \geq I(U;W)$
\end{corollary}

\begin{proof}
  \[
    I(U;VW) = I(U;V) + I(U;W|V) = I(U;V)
  \]
  and
  \[
    I(U;VW) = I(U;W) + I(U;V|W) \geq I(U;W)
  \]
\end{proof}

\begin{theorem}
  Given $U$ a RV taking values in $\U$ then $0 \leq H(U) \leq \log | \U |$. $H(U)=0$ iff $U$ is constant, $H(U)=\log | \U |$ iff $U$ is $p(u) = 1 / |\U|$ for all $u$.
\end{theorem}

\begin{proof}
For the lower bound,
  \[
    H(U) = \sum_u \underbrace{p(u)}_{\geq 0} \underbrace{\log \frac 1 {p(u)}}_{\geq 0} \geq 0
  \]
For the upper bound,

\begin{align*}
  H(U) - \log | \U |
  &= \sum_u p(u) \log \frac 1 {p(u)} - \sum_u p(u) \log |\U|\\
  &= \frac 1 {\ln 2} \sum_u p(u) \ln \frac 1 {|\U | p(u)}\\
  &\leq \frac 1 {\ln 2} \sum_u p(u) \left(\frac 1 {|\U | p(u)} - 1 \right)\\
  &=\frac 1 {\ln 2} \left[ \sum_u \frac 1 {|\U |} - \sum_u p(u) \right]\\
  &=0
\end{align*}
\end{proof}

\begin{theorem}
  $I(U;V) = 0 \iff U \bot V$
\end{theorem}


\begin{definition}[Entropy rate of a stochastic process]
  $\lim_{n\to \infty} \frac 1 n H(U^n)$ if the limit exists.
\end{definition}

\begin{theorem}
  For stationary stochastic process $U^n$, the sequences
  \[
    a_n = \frac 1 n H(U^n) \text{ and } b_n = H(U_n|U^{n-1})
  \]
  are positive and non increasing. Then $a=\lim_{n\to \infty} a_n$ and $b=\lim_{n\to \infty} b_n$ exists and $a=b$.
\end{theorem}

\begin{proof}
\begin{align*}
  b_{n+1}
  &= H(U_{n+1}|U_1, U_2,\dots,U_n)\\
  &\leq H(U_{n+1}|U_2,\dots,U_n)\\
  &= H(U_n|U_1, U_2,\dots,U_{n-1})\\
  &= b_n \text{ , because $U_1 \dots U_n \sim U_2 \dots U_{n+1}$ (Stationarity).}
\end{align*}


Hence, it is non-increasing.\\\\

For the \{$a_n$\}, observe that

\begin{align*}
a_n = \frac{1}{n} H(U^n)
&= \frac{1}{n} \bigg[ H(U_1)+H(U_2|U_1)+H(U_3|U^2) +\dots+H(U_n|U^{n-1}) \bigg]\\
&= \frac{1}{n} \bigg[ b_1+b_2+\dots+b_n\bigg]
\end{align*}


and by the "Lemma", whenever $b_n \rightarrow b$ , \space $a_n \rightarrow b$
\end{proof}
\begin{lemma}[Cesaro]
	Suppose $b_n \rightarrow b$, \\\\

	then,
	\[
	a_n = \frac{1}{n} \bigg[ b_1+b_2+\dots+b_n\bigg] \text{ also converges and to 1.}
	\]
\end{lemma}
\begin{proof}
	Since $b_n \rightarrow b$ , $\bigg( \equiv \forall \epsilon > 0$ , $\exists$ $n(\epsilon)$ s.t $ \forall n > n(\epsilon)$  $ |b_n-b| < \epsilon\bigg)$\\

	$\exists B $ s.t. $ |b_n| < B$ for all n.\\\\
	Take $n > n_1(\epsilon) \triangleq \dots $ then
	\[
		|a_n-b| \leq \frac{|b_1-b|+|b_2-b|+|b_3-b|+\dots+|b_n-b|}{n}
	\]
	\[
		\text{so  }  |a_n-b| \leq \frac{1}{n}\bigg[ \sum_{i=1}^{n_0(\epsilon)} \underbrace{|b_i-b|}_{2B}  + \sum_{i=n_0(\epsilon)+1}^{n} \underbrace{|b_i-b|}_{\leq \epsilon}\bigg] \leq \frac{n_0(\epsilon) 2B}{n} + \epsilon < 2\epsilon
	\]
	\[
		\text{for } n>n_1(\epsilon) \triangleq \text{max, } \{ n_0(\epsilon) \frac{1}{\epsilon} n_0(\epsilon) 2B \}
	\]
\end{proof}



% -------------------------------------------------------------------------
% class 9.10.2017

\begin{theorem}
  Given a stationary process with entropy rate $r$:
  \begin{align*}
    r = \lim_{n \rightarrow \infty} \frac{1}{n} H(\U^n)
  \end{align*}

  then
  \begin{enumerate}
    \item for every source coding scheme
    \begin{align*}
      \C_n: \U^n \rightarrow \{0, 1\}^*
    \end{align*}
    the expected number of bits / letter is given by
    \begin{align*}
      \frac{1}{n} E[l(\C(\U^n))] \geq r
    \end{align*}
    \item for any $\epsilon > 0$, there exists a source coding scheme $\C_n: \U^n \rightarrow \{0, 1\}^*$ s.t.
    \begin{align*}
      \frac{1}{n} E[l(\C_n(\U^n))] < r + \epsilon
    \end{align*}
  \end{enumerate}
\end{theorem}

\begin{proof}
  \begin{enumerate}
    \item we already know
    \begin{align*}
      \frac{1}{n} E[l(\C_n(\U^n))] \geq \frac{1}{n} H(\U_1 ... \U_n)
    \end{align*}
    and the right term is decreasing
    \item we also know that for each $n, \exists \C_n$ that is prefix-free s.t.
    \begin{align*}
      E[l(\C_n(U^n))] < \underbrace{\frac{1}{n} H(\U^n)]}_{r} + \underbrace{\frac{1}{n}}_{0}
    \end{align*}
    we can find $n$ large enough s.t. the RHS $< r + \epsilon$
  \end{enumerate}
\end{proof}

% -------------------------------------------------------------------------
% -------------------------------------------------------------------------

\section{Typicality and typical set}

Suppose we have a sequence $U_1, U_2, ...$ of i.i.d. random variables taking values in an alphabet $\U$.
Suppose we observe $u_1,u_2..., u_n$. We will call it to be \textit{typical-$(\epsilon, p)$} if
\begin{align*}
  p(u) (1 - \epsilon)
  \leq \frac{\# \text{ of times $u$ appears in $u_1, ..., u_n$}}{n}
  \leq p(u)(1+\epsilon)
\end{align*}

\begin{theorem}
  $u^n$ is $(\epsilon, p)$-typical then
  \begin{align*}
    2^{-n H(u)(1 + \epsilon)}
    \leq Pr(U^n = u^n)
    \leq 2^{-n H(u)(1 + \epsilon)}
  \end{align*}
\end{theorem}

\begin{proof}
  \begin{align*}
    Pr(U^n = u^n) &= \prod_{i=1}^n Pr(U_i = u_i) = \prod_{i=1}^n p(u_i) = \prod_{u \in U} p(u)^{\#_u}
  \end{align*}
  with $\#_u$ the number of times $u$ appears in $u_1, ..., u_n$ where
  \begin{align*}
    n (1-\epsilon) p(u) \leq \#_u \leq n(1+\epsilon)p(u)
  \end{align*}
  consequently
  \begin{align*}
    p(u)^{(n p(u)(1-\epsilon))} \geq p(u)^{\#_u} \geq p(u)^{n p(u)(1+\epsilon)}
  \end{align*}
  then
  \begin{align*}
    (\prod_{n} p(u)^{p(u)})^{(1-\epsilon)n}
    \geq Pr(U^n = u^n)
    \geq (\prod_{n} p(u)^{p(u)})^{(1+\epsilon)n}
  \end{align*}
  but
  \begin{align*}
    p(u)^{p(u)} = 2^{-p(u) \log(\frac{1}{p(u)})} \Rightarrow \prod p(u)^{p(u)} = 2^{-H(u)}
  \end{align*}
\end{proof}

\begin{definition}[Typical set]
  \begin{align*}
    T(n, \epsilon, p) = \{ u^n \in U^n : u^n \text{ is } (\epsilon, p)\text{-typical}\}
  \end{align*}
\end{definition}

\begin{theorem}
  \begin{enumerate}
    \item if $u^n \in T(n, \epsilon, p)$ then
    \begin{align*}
      p(u^n) = Pr(U^n = u^n) = 2^{-n H(u)(1 \pm \epsilon)}
    \end{align*}
    when $U_i$ i.i.d.
    \item
    \begin{align*}
      \lim_{n \rightarrow \infty} Pr(U^n \in T(n, \epsilon, p)) = 1
    \end{align*}
    \item
    \begin{align*}
      |T(n, \epsilon, p)| \leq 2^{n (H(u)(1 + \epsilon))}
    \end{align*}
    \item
    \begin{align*}
      |T(n, \epsilon, p)| \geq (1-\epsilon) 2^{n H(u)(1-\epsilon)}
    \end{align*}
  \end{enumerate}
\end{theorem}

\begin{proof}
  \todo{}
\end{proof}


\begin{definition}[Kullback-Leiber divergence (information gain)]
  If we compress data in a manner that assumes $q(u)$ is the distribution underlying some data, when, in reality, $p(u)$ is the correct distribution, the Kullback-Leiber divergence is the number of average additional bits per datum necessary for compression.
\end{definition}





% ----------------------------- %
% ---- lecture 2017-10-10 ----- %
% ------------ jb ------------- %
% ----------------------------- %

\begin{lemma}
  if $U_1 \dots U_n$ are i.i.d. with distribution $q$ and $u_1 \dots u_n$ is $(\epsilon, p)$-tipycal, then

  \begin{align*}
    \Pr{U^n = u^n}
    &= \left[ \prod q(u)^{p(u)}\right]^{n (1 + \epsilon)} \\
    &= 2^{-n(1\pm \epsilon)} \sum_u p(u) \log \frac 1 {q(u)}
  \end{align*}
\end{lemma}

We know that $\Pr{U^n \in T(n, \epsilon, p)} \to 1$ as $n\to \infty$ and

\[
  (1-\epsilon)2^{nH(U)(1-\epsilon)} \leq | T(n, \epsilon, p) | \leq 2^{nH(U)(1+\epsilon)}
\]
\todo{Something wrong in the formula above}

\begin{observation}

Suppose $U_1 \dots U_n$ are i.i.d. following $q$ and $u^n \in T(n, \epsilon, p)$

\[
  \left[ \prod_u q(u)^{p(u)} \right]^{n(1+\epsilon)} \leq \Pr{U^n = u^n} \leq  \left[ \prod_u q(u)^{p(u)} \right]^{n(1-\epsilon)}
\]

and
\[
  \prod_u q(u)^{p(u)} = 2^{-\sum p(u) \log \frac 1 {q(u)}}
\]

\[
  \sum_u p(u) \log \frac 1 {q(u)} =
  \underbrace{\sum_u p(u) \log \frac 1 {p(u)}}_{H(p)} +
  \underbrace{\sum_u p(u) \log \frac {p(u)} {q(u)}}_{D(p||q)}
\]
\end{observation}

\begin{corollary}
  if $U_1 \dots U_n$ are i.i.d. following distribution $q$, then
  \[
    2^{-n[(1+\epsilon)D(p||q)+2\epsilon H(p)]}
    \leq
    \Pr{U^n \in T(n,\epsilon,p)}
    \leq
    2^{-n[(1-\epsilon)D(p||q) - 2\epsilon H(p)]}
  \]
\end{corollary}

\begin{proof}
  \[
    \Pr{U^n \in T(n,\epsilon,p)} = \sum_{u^n \in T(n, \epsilon, p)} \Pr{U^n = u^n}
  \]
  We have
  \begin{align*}
    2^{-n[H(p) + D(p||q)](1+\epsilon)}
    \leq
    &\Pr{U^n = u^n}
    \leq
    2^{-n[H(p) + D(p||q)](1-\epsilon)}\\
    2^{nH(p)(1-\epsilon)}
    \leq
    &|T(n, \epsilon, p)|
    \leq
    2^{nH(p)(1+\epsilon)}
  \end{align*}
\end{proof}


\begin{observation}
  \[
    D(p||q) = \sum_u p(u) \log \pfrac {p(u)} {q(u)} \geq 0 \text{ with equality iff } p = q
  \]
\end{observation}


\begin{example}
  $U \in \{0,1\}$, $p=\frac 1 2, \frac 1 2$, $q=\frac 1 2 - \delta, \frac 1 2 + \delta$
  \[
    D(p||q) = \frac 1 2 \log \frac 1 {1-2\delta} + frac 1 2 \log \frac 1 {1+2\delta} = \frac 1 2 \log \frac 1 {1-4\delta^2} = - \frac 1 2 \log (1-4\delta^2) \approx \frac 1 2 4\delta^2 + o(\delta^4)
  \]
  So if we want $2^{-nD(p||q)}$ small $n=\Omega(1/\delta^2)$
\end{example}

\begin{example}
  Suppose we are told that $U$ is $p$ distributed and $p(u)$ are powers of 2. We design a prefix-free code $\C$ to minimize $\sum_u p(u) l(\C(u))$. We have been misinformed and $U\sim q$, then:

  \begin{align*}
    \Ex{l(\C(u))}
    &= \sum_u q(u) \log \frac 1 {p(u)}\\
    &= \underbrace{H(q)}_{\text{length for optimal code}} + \underbrace{D(q||p)}_{\text{penalty for misbelief}}
  \end{align*}
\end{example}

\subsection{Universal data compression}

Suppose we know that the distribution $p$ of $U$ is either $p_1$, $p_2$ ... $p_k$, can we design a code $\C: U \to \{0,1\}^*$

\[
  \Ex{l(\C(U))} \leq H(U) + \text{small for every } p
\]

\[
  \Ex{\frac 1 n l(\C(U))} \leq o(n) + \Ex{h_2 \pfrac K n}
\]
with $K = \sum_{i=1}^n u_i$

We have $\frac {\Ex{K}} n = \theta_1$ and $\Ex{h_2\pfrac K n} \leq h_2 \left(\Ex{\frac K n} \right) = h_2(\theta)$

\paragraph{Design $\C$}
Because the probability of a bit string is only dependant of the number of
1s (or 0s), it makes sense to encode two strings with the same numbers of 1
with code words of same lengths.
Given $u_1 \dots u_n \in \{0,1\}^n$, first count the number of 1, call it $k$.


\[
  \C(u_1 \dots u_n) =
  \underbrace{\text{describe } k}_{\lceil \log (n+1)\rceil}
  \underbrace{\text{describe } u_1 \dots u_n}_{\lceil \log {n \choose k}\rceil}
\]

We now want to evaluate

\[
  \frac 1 n \Ex{l(\C(U))}
\]

when $U_1 \dots U_n$ are i.i.d with $p_1 = \theta$ and $p_0 = 1 - p_1$

\begin{observation}

for any $0 \leq \alpha \leq 1$

\begin{align*}
  1 = 1^n = (\alpha + (1-\alpha))^n &= \sum_{i=0}^n {n \choose i} \alpha^i (1-\alpha)^{k-i}\\
  & \geq {n \choose k} \alpha^k (1-\alpha)^{n-k}
\end{align*}

Then for all $\alpha$

\[
{n \choose k} \leq \alpha^{-k}(1-\alpha)^{-(n-k)} = 2^{-n(\frac k n \log \frac 1 \alpha + (1-\frac k n) \log \frac 1 {1-\alpha})}
\]

We pick $\alpha = \frac k n$, and we get

\[
  {n \choose k} < 2^{n h_2 \pfrac k n}
\]

\end{observation}

Using this bound we have

\[
  \frac 1 n l(\C(u_1 \dots u_n)) \leq \frac 2 n + \frac {\log (n+1)} n + h_2\pfrac k n
\]

\[
  \Ex{\frac 1 n l(\C(U))} \ leq o(n) + \Ex{h_2 \pfrac k n}
\]

\begin{claim}
Suppose $U_i$ are i.i.d. with $\Pr{U_1=1}=\theta$. We have $\Ex{ \frac k n} = \theta$ and $\Ex{h_2 \pfrac k n } \leq h_2(\Ex{\frac k n}) = h_2(\theta)$. So
\[
  \lim_{n\to \infty} \frac 1 n \Ex{l(\C(u_1\dots u_n))} \leq h_2(\theta)
\]
consequently this scheme is asymptotically optimal.
\end{claim}

\begin{proof}
  To prove the claim we need to show that if $\beta_1\dots \beta_k$ are in $[0,1]$ and $q_1 \dots q_k$ are non negative numbers that sum to 1 then

  \[
    \sum_{i=1}^k q_i h_2(\beta_i) \leq h_2 \left(\sum_{i=1}^k q_i \beta_i \right)
  \]

  Let $U$ and $V$ be random variables with $U \in \{0,1\}$ and $V \in \{1,\dots,k\}$ with

  \begin{align*}
    \Pr{V=i} &= q_i\\
    \Pr{U=1|V=i} &= \beta_i\\
    \Pr{U=0|V=i} &= 1- \beta_i\\
  \end{align*}
  Then,
  \begin{align*}
    \Pr{U=1} &= \sum_i q_i \beta_i\\
    H(U) &= h_2\left(\sum_i q_i \beta_i \right)\\
    H(U|V) &= \sum_i q_i h_2(\beta_i)
  \end{align*}

  And we already know that $H(U) \geq H(U|V)$
\end{proof}



% ----------------------------- %
% ---- lecture 2017-10-16 ----- %
% --------- thomas ------------ %
% ----------------------------- %









% -------------------------------------------------------------------------
% lesson - 16.0.2017 - Thomas

\todo{THOMAS SCRIBE ICI}

% -------------------------------------------------------------------------
% lesson - 17.0.2017 - sebastien

Suppose we have an infinite string $u_1 u_2 ..., u_i \in U$, and 
$$u_1 u_2 ... = v_1 v_2 ... \text{ with } v_i \in U^*, v_i \neq v_j \text{ when } i \neq j$$
for any $k$ we have
\begin{align*}
  \lim_{m \to \infty} \frac{length(v_1...v_m)}{m} \geq k
  \Rightarrow \lim_{m \to \infty} \frac{length(v_1 ... v_m)}{m} = 
  \infty
\end{align*}

\begin{definition}
  Given an infinite string $u_1 u_2 ...$ and a machine $M$, let
  \begin{align*}
    \rho_{M}(u_1 u_2 ...) = \overline{\lim_{n \to \infty}} \frac{\text{length of the output } M \text{ after reading } u_1 u_2...}{n}
  \end{align*}
  also given $s > 0$, define
  \begin{itemize}
    \item The compressibility of $U^*$ be s-state machines
    \begin{align*}
      \rho_s (u_1 u_2 ...) = \min_{M} \rho_{M}(u_1 u_2 ...)
    \end{align*}
    with $M$ an $s'$-state machine with $s' \leq s$
    \item Compressibility of $U^*$ by finite state machines
    \begin{align*}
      \rho_{FSM} (u_1 u_2 ...) = \lim_{s \to \infty} \rho_s (u_1 u_2 ...)
    \end{align*}
  \end{itemize}
\end{definition}

\begin{definition}
  Suppose $u_1 u_2 ...$ an infinite sequence, define $m(n)$ as the largest $m$ for which $u_1 ... u_n = v_1 ... v_m$ with distinct $v_1 ... v_m$
  \begin{example}
    $$ u = aaaaaaaaa, \quad \underbrace{\emptyset}_{v_1} \underbrace{a}_{v_2} \underbrace{aa}_{v_3} \underbrace{aaa}_{v_4} \underbrace{aaaa}_{v_5} \quad \Rightarrow m(10) = 5 $$
  \end{example}
\end{definition}

So far we know that 

\begin{align*}
  \frac{\text{ length of the output of any s-state IL machine when it reads } u_1 u_2 ... }{n} \geq \frac{m(n) \log(\frac{m(n)}{8 s^2})}{n}
\end{align*}
with 
\begin{align*}
  \frac{m(n) \log(\frac{m(n)}{8 s^2})}{n} = \frac{m(n) \log(m(n))}{n} - \frac{m(n) \log(8 s^2)}{length(v_1 ... v_m)}
\end{align*}

hence if $M$ is a s-state machine
\begin{align*}
  \rho_M (u_1 u_2 ...) \geq \overline{\lim_{n \to \infty}} \frac{m(n) \log(m(n))}{n} 
  \quad \text{ then } \quad 
  \rho_{FSM} (u_1 u_2 ...) \geq \overline{\lim_{n \to \infty}} \frac{m(n) \log m(n)}{n}
\end{align*}


% -------------------------------------------------------------------------
% Lemple-Ziv

\newpage
\section{Lemple-Ziv data compression method}

Given some alphabet $U$ to both encoder and decoder, they also agree an order on $U$:

\begin{enumerate}
  \item Start with a dictionary $D = U$
  \item To each word $w \in D$, assign a $\lceil \log |D| \rceil $-bit binary description in the dictionary order
  \item Parse the first word in $u_1 u_2 ...$ in the dictionary, output its binary description
  \item replace $w$ in $D$ by $\{ wu, \forall u \in U \}$.
  \item Go to 2.
\end{enumerate}

\begin{example}
  Define an alphabet $U = \{a, b, c\}$ with $a \leq b \leq c$ and an input message
  $$ u = b b a c a c $$
  \begin{itemize}
    \item Create the dictionary $D = \{a, b, c\}$ and its corresponding binary description $D_{bin} = \{00, 01, 10\}$
    \item The first word in the message is $'a'$, output its binary description
    $$ output = 01 $$ 
    \item Update the dictionary: 
    $$ D = \{a, ba, bb, bc , c\} \quad D_{bin} = \{000, 001, 010, 011, 100\} $$
    \item Parse the next word $'ba'$ and output its binary description
    $$ output = 01 001 $$ 
    \item Update the dictionary
    $$ D = \{a, baa, bab, bac, bb, bc , c\} \quad D_{bin} = \{000, 001, ...\} $$
    \item Continue until the end of the input data...
  \end{itemize}
  The decoder can proceed in a similar way to iteritavely update the dictionary while decoding the message.
\end{example}

\subsection{Analysis of LZ}

Observe that LZ parses the string $u_1 u_2 ...$ into $v_1 v_2 ...$ with $v_i \in U^*$ or $v_i \in D_i$ where $D_i$ is the dictionary at step $i$.

When going from iteration $i \rightarrow i+1$, $v_i$ is removed from $D$, consequently $v_1, v_2, v_3$ are distinct.

The length of the output of LZ after reading $v_1 ... v_m$ is given by
\begin{align*}
  \text{LZ output's length} = \lceil \log |U| \rceil + \lceil \log (2 |U| - 1) \rceil
  + \lceil \log (3 |U| - 2) \rceil + ... 
  + \lceil \log (m|U| - m + 1) \rceil
\end{align*}
we observe that
\begin{align*}
  \text{LZ output's length} < m(\log(m |U|) + 1) = m \log(2 m |U|)
\end{align*}

Also we have

\begin{align*}
  \text{\# bits / letter} &< \frac{m \log(2m |U|)}{length(v_1 ... v_m)} \\
  &= \frac{m \log(m)}{ length(v_1 ... v_m)} + \frac{m \log(2 |U|)}{length(v_1 ... v_m)}
\end{align*}
therefore
\begin{align*}
  \rho_{LZ}(u_1 u_2 ...) = \lim_{m \to \infty} \frac{\text{\# bits}}{\text{letter}} \leq \lim_{m \to \infty} \frac{m \log(m)}{length(v_1 ... v_m)} \leq \lim_{n \to \infty} \frac{m(n) \log(m(n))}{n} \leq \rho_{FSM}(u_1 u_2 ...)
\end{align*}

So we have proved the following theorem:

\begin{theorem}
  for every $u_1 u_2 ...$
  $$ \rho_{LZ}(u_1 u_2 ...) \leq \rho_{FSM}(u_1 u_2 ...) $$
\end{theorem}

\begin{corollary}
  if $u_1 u_2 ...$ is stationary
  $$ \rho_{LZ}(u_1 u_2 ...) = \text{ entropy rate of } u_1 u_2 ...$$
\end{corollary}


% -------------------------------------------------------------------------
% appendices


\newpage
\begin{appendices}
\section{Markov chains}
\label{appendix:markov-chains}

$U_1 - U_2 - \dots - U_n$ forms a Markov chain if the joint probability
distribution of the RVs is
\[
  p(a,b,c,d) = p(a)p(b|a)p(c|b)p(d|c)
\]
which is equivalent to $(U_1, \dots, U_{k-1})$ are independant of $(U_{k+1}, \dots, U_n)$ when conditionned on $U_k$ for any $k$.


\begin{theorem}
  The reverse of a MC is a MC
\end{theorem}


\section{Stochastic processes}
\label{appendix:stoch-proc}

A stochastic process is a collection $U_1, U_2 \dots U_n$ of RVs each taking values in $\U$. It is described by its joint probability
\[
  p(u^n) = P(U_1 \dots U_n = u_1 \dots u_n) = P(U^n = u^n)
\]

\begin{definition}[Stationary stochastic process]
  A process $U_1, U_2, \dots$ is called stationary if for every $n$ and $k$ and $u_1 \dots u_n$, we have
  \[
    p(u^n) = p(U_1 \dots U_n = u_1 \dots u_n) = p(U_{1+k} \dots U_{n+k} = u_1 \dots u_n)
  \]
  In other words, the process is time shift invariant.
\end{definition}

\end{appendices}


\end{document}
